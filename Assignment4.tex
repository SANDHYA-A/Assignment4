\documentclass[journal,12pt,twocolumn]{IEEEtran}
\IEEEoverridecommandlockouts
\usepackage{cite}
\usepackage{amsmath,amssymb,amsfonts,bm}
\usepackage{mathtools}
\usepackage{tkz-euclide} 
\usepackage{tikz}
\usetikzlibrary{calc,math}
 \usepackage{caption}
\usepackage{listings}
\let\vec\mathbf
\numberwithin{equation}{subsection}

\newcommand{\myvec}[1]{\ensuremath{\begin{pmatrix}#1\end{pmatrix}}}
\newcommand{\norm}[1]{\left\lVert#1\right\rVert}

\renewcommand\thesection{\arabic{section}}
\renewcommand\thesubsection{\thesection.\arabic{subsection}}
\renewcommand\thesubsubsection{\thesubsection.\arabic{subsubsection}}

\renewcommand\thesectiondis{\arabic{section}}
\renewcommand\thesubsectiondis{\thesectiondis.\arabic{subsection}}
\renewcommand\thesubsubsectiondis{\thesubsectiondis.\arabic{subsubsection}}
%\renewcommand{\theequation}{\theenumi}
%\numberwithin{equation}{enumi}

\providecommand{\mbf}{\mathbf}
\providecommand{\pr}[1]{\ensuremath{\Pr\left(#1\right)}}
\providecommand{\qfunc}[1]{\ensuremath{Q\left(#1\right)}}
\providecommand{\sbrak}[1]{\ensuremath{{}\left[#1\right]}}
\providecommand{\lsbrak}[1]{\ensuremath{{}\left[#1\right.}}
\providecommand{\rsbrak}[1]{\ensuremath{{}\left.#1\right]}}
\providecommand{\brak}[1]{\ensuremath{\left(#1\right)}}
\providecommand{\lbrak}[1]{\ensuremath{\left(#1\right.}}
\providecommand{\rbrak}[1]{\ensuremath{\left.#1\right)}}
\providecommand{\cbrak}[1]{\ensuremath{\left\{#1\right\}}}
\providecommand{\lcbrak}[1]{\ensuremath{\left\{#1\right.}}
\providecommand{\rcbrak}[1]{\ensuremath{\left.#1\right\}}}

\lstset{
frame=single, 
breaklines=true,
columns=fullflexible
}

\begin{document}

\title{Matrix Theory EE5609 - Assignment 4\\
}

\author{\IEEEauthorblockN{Sandhya Addetla}\\
\IEEEauthorblockA{PhD Artificial Inteligence Department} \\
AI20RESCH14001\\
 }

\maketitle
\begin{abstract}
 This document finds the point at which a line touches a circle.

\end{abstract}

Download all python codes from 
\begin{lstlisting}
https://github.com/SANDHYA-A/Assignment4/blob/master/Assignment4.py
\end{lstlisting}

\section{Problem Statement}
 For what values of $m$ does the line 
\begin{align}
\myvec{m &-1}\vec{x}=0 \label{1.1}
\end{align}
touch the circle
\begin{align}
\vec{x}^T\vec{x}-\myvec{6 & 2}\vec{x}+8 = 0 \label{1.2}
\end{align}

\section{Solution}
The general equation of a circle is 
\begin{align}
\implies \vec{x}^T\vec{x}+ 2\vec{u}^T\vec{x} + f = 0 \label{2.1}
\end{align}
\begin{align}
\text{If $r$ is radius,} f =\vec{u}^T\vec{u}-r^2  \label{2.2} \\  
  \text{center }\vec{c} =-\vec{u}
\end{align}
From equations \ref{1.2} and \ref{2.1},
\begin{align}
 \vec{u}  = \myvec{-3 \\ -1}\\
\implies \text{Center of the cirlce } \vec{c} = \myvec{3 \\ 1}
\end{align}
Also, radius can be determined as follows from \ref{2.2}
\begin{align}
  \implies 8=\myvec{-3&-1}\myvec{-3\\-1}-r^2\\
  \implies 8=10 -r^2\\
  \implies r=\sqrt{2}
\end{align}
Given equation of the line is
\begin{align}
\myvec{m&-1}\vec{x}&=0 
\end{align}

It can be expressed as:-
\begin{align}
L: \quad \vec{x} = \vec{q} + \mu \vec{m} \quad \mu \in \mathbb{R} \label{2.10} \\
\end{align}
The normal vector to the line is obtained as
\begin{align}
\vec{n} = \vec{q} + \vec{u}\\
\implies \vec{q}  = \vec{n} -  \vec{u}\\
\vec{n} =\myvec{m & -1 }^T  \text{ and } \vec{u} =\myvec{-3 & -1}^T\\
\vec{q} = \myvec{m+3 \\ 0}
\end{align}
The point $\vec{q}$ also satisfies the equation of the circle at \ref{1.2}.
\begin{align}
\vec{q}^T\vec{q}+ 2\vec{u}^T\vec{q} + f = 0 \\
%\vec{q}^T\vec{q}-\myvec{6 &  2}\vec{q}+8 = 0 
\end{align}
\begin{multline}
(\vec{n} -\vec{ u})^T(\vec{n} -\vec{ u})+2\vec{u}^T(\vec{n} -\vec{ u})\\+f = 0
\end{multline}
\begin{multline}
\norm{\vec{n}}^2 - \vec{n}^T \vec{u} - \vec{u}^T\vec{n} + \norm{\vec{u}}^2  + 2\vec{u}^T\vec{n}\\ -2 \norm{\vec{u}}^2 +f = 0 
\end{multline}
\begin{multline}
\norm{\vec{n}}^2 - \vec{n}^T \vec{u} +\vec{u}^T\vec{n} - \norm{\vec{u}}^2  +f = 0 
\end{multline}
\begin{multline}
(\vec{m}^2 +1) - (-3\vec{m} + 1) +\\(-3\vec{m} + 1) - 10 +8 = 0 
\end{multline}
\begin{align}
m^2 -1 =0\\
m=\pm 1
\end{align}
For the line \ref{1.1} to be a tangent to circle at equation \ref{1.2}, the values of $m$ are $\pm1$
\begin{figure}[!]
\includegraphics[width=1\columnwidth]{tangent.jpg}
\caption{Circle with tangent}
\end{figure}


\end{document}